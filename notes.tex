\documentclass{scrartcl}
\usepackage[utf8]{inputenc}
\usepackage[ngerman]{babel}
\usepackage[fleqn]{amsmath}
\usepackage{amssymb}
\usepackage{parskip}
\usepackage{graphicx}

\usepackage{listings}
\lstset{language=Octave, basicstyle=\tt, tabsize=8,
  breaklines=true, caption=\texttt\lstname, captionpos=b}
\DeclareFontShape{OT1}{cmtt}{bx}{n}
{<5><6><7><8><9><10><10.95><12><14.4><17.28><20.74><24.88>cmttb10}{}

\begin{document}

\title{Numerische Mathematik UE -- 2. Projekt, Teilprojekt 1 und 2}
\author{Gabriel Ebner, Johannes Hafner}
\maketitle

\section{Theoretischer Teil}

\subsection{Singulärwertzerlegung}

Bla bla bla

\subsection{Eigenwertzerlegung}

Bla bla bla

\subsection{Vandermondematrix}

Bla bla bla

\section{Experimenteller Teil}

\subsection{Plots der Fehlerverläufe und Konditionen}

\begin{figure}[!htb]
\centering
\includegraphics{fig_eig.pdf}
\caption{Matrix in Eigenwertzerlung: Fehler beim Lösen des LSG, Kondition von \(X \Lambda X^{-1}\), Kondition von \(X\). }
\label{fig:eig}
\end{figure}

\begin{figure}[!htb]
\centering
\includegraphics{fig_svd.pdf}
\caption{Matrix in Singulärwertzerlung: Fehler beim Lösen des LSG, Kondition von \(UDV^T\). }
\label{fig:svd}
\end{figure}

\begin{figure}[!htb]
\centering
\includegraphics{fig_vander.pdf}
\caption{Vandermondematrix: Fehler beim Lösen des LSG, Kondition von \(A\). }
\label{fig:svd}
\end{figure}

\subsection{Betriebsmittel}

Die Berechnungen wurden mit double precision unter Octave 3.2.4 und Ubuntu
Oneiric (amd64) auf einem AMD Phenom II X4 965 Prozessor durchgeführt.

\subsubsection{Quelltext}

\lstinputlisting{eigm.m}
\lstinputlisting{eigv.m}
\lstinputlisting{svdm.m}
\lstinputlisting{svdv.m}
\lstinputlisting{vanderm.m}
\lstinputlisting{vanderv.m}
\lstinputlisting{errsolve.m}
\lstinputlisting{ploterrs.m}
\lstinputlisting{ploterrs_eig.m}
\lstinputlisting{plots.m}

\end{document}
